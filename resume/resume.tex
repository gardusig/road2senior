\documentclass[a4paper,10pt]{article}
\usepackage[a4paper,margin=0.66in]{geometry}
\usepackage{enumitem}
\usepackage{sourcesanspro}
\usepackage{hyperref}
\usepackage{titlesec}
\usepackage{parskip}
\linespread{0.96}
\pagestyle{empty}

% Define title format
\titleformat{\section}{\Large\bfseries}{}{0em}{}[\titlerule]
\titlespacing*{\section}{0pt}{1.5em}{0.5em}

% Set font family
\renewcommand{\familydefault}{\sfdefault}

% Reduce space between list items
\setlist[itemize]{itemsep=0pt, topsep=0pt}

% Begin document
\begin{document}

% Name and contact information
\begin{center}
    {\LARGE \textbf{Gustavo Gardusi}}\\
    \vspace{0.5em}
    \href{mailto:gustavo.gardusi@gmail.com}{gustavo.gardusi@gmail.com} \hspace{1cm} 
    \href{https://github.com/gardusig}{github.com/gardusig}
\end{center}


% Work Experience
\section*{Work Experience}

\textbf{Amazon Web Services (AWS)} \hfill São Paulo, São Paulo, Brazil \\
Software Development Engineer II \hfill \textit{Nov 2024 - Present}
\begin{itemize}
    \item Developed and optimized a system as a full-stack engineer using React (TypeScript) for the front-end and Java/Python (API Gateway + Lambda) for the back-end, storing data in DynamoDB and managing infrastructure with AWS CDK.
    \item Implemented key components for the Skill Builder public profile, transitioning Figma designs into a production-ready Next.js and Tailwind CSS implementation.
\end{itemize}

\textbf{Orkes} \hfill Cupertino, California, United States (remote from Brazil) \\
Senior Software Engineer\hfill \textit{Jan 2022 - Jul 2023}
\begin{itemize}
    \item Led the initiative for polyglot clients by creating an open-source collection of projects enabling users to scale up multiple Netflix Conductor workflow task workers concurrently. Designed a synchronous workflow execution and resolved message queue acknowledgment issues on batch requests.
    \item Optimized resource allocation by making batch requests at clients while managing active threads, reducing workflow processing time by nearly 20\%. Quickly identified and fixed compatibility issues to ensure project continuity, earning the trust of the largest company customer.
    \item Established a foundation for all end-to-end tests and increased the test coverage to nearly 100\% of relevant code. Pioneered the creation of Python and Go SDKs and contributed 40,000 lines of code to the Java SDK. Co-creator of C\# and JS SDKs. Previously, only Java was fully supported.
\end{itemize}

\textbf{Amazon.com.br} \hfill São Paulo, São Paulo, Brazil \\
Software Development Engineer II \hfill \textit{Apr 2021 - Jan 2022}
\begin{itemize}
    \item Improved the resilience score of two critical Java services responsible for invoice generation by adding a new load testing approval step, which fails if minimum transaction rates or maximum latency requirements are not met.
    \item Expanded the Fulfillment By Amazon program in Brazil by managing diverse invoice expiration and cancellation rules, enabling sellers from multiple states.
    \item Leveraged historical metrics data to calculate an average distribution of requests and identify the expected load a single host could handle, ensuring adequate scaling and avoiding overloads or unnecessary costs during Black Friday 2021.
\end{itemize}

\textbf{Beyond} \hfill São Paulo, São Paulo, Brazil \\
Software Engineer \hfill \textit{Aug 2019 - Apr 2021}
\begin{itemize}
    \item Spearheaded the development of a high-frequency trading web application certified by B3, using C++ and WebSocket for communication, implementing an efficient order management system using the Financial Information eXchange protocol. Handled millions of orders per day and generated thousands in revenue.
    \item Created numerous APIs in Java to handle stock market data, complemented by Python scripts for data parsing and filtering, resulting in a trading strategy simulator.
\end{itemize}

\textbf{Algar Telecom} \hfill Uberlândia, Minas Gerais, Brazil \\
Software Engineer \hfill \textit{Sep 2017 - Aug 2019}
\begin{itemize}
    \item Developed an automation application in Python to collect data from 300 routers and configure them in real-time, storing data in MySQL databases linked to Metabase for generating dashboards and reports.
\end{itemize}

% Awards
\section*{Awards (Competitive Programming)}
\begin{itemize}
    \item \textbf{Meta HackerCup:} Earned 3 consecutive T-shirts, awarded to the top 2,000 participants. Best rank: 1,250th.
    \item \textbf{Codeforces/CodeChef:} Achieved First Division status on both platforms, placing in the top 10\%.
    \item \textbf{ICPC Latin America Finals:} Competed in 4 regional finals. Best rank: 16th out of 1,000 teams.
\end{itemize}

% Academic Background
% \section*{Academic Background}
% \textbf{Federal University of Uberlandia (UFU)} \hfill Uberlândia, Minas Gerais, Brazil \\
% Computer Science \hfill \textit{January 2014 - December 2018}
% \begin{itemize}
%     \item Completed over 60\% of the Computer Science full course curriculum, emphasizing competitive programming.
% \end{itemize}

\end{document}
