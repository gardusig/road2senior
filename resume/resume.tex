\documentclass[a4paper,10pt]{article}
\usepackage[a4paper,margin=0.66in]{geometry}
\usepackage{enumitem}
\usepackage{sourcesanspro}
\usepackage{hyperref}
\usepackage{titlesec}
\usepackage{parskip}
\linespread{0.96}
\pagestyle{empty}

% Define title format
\titleformat{\section}{\Large\bfseries}{}{0em}{}[\titlerule]
\titlespacing*{\section}{0pt}{1.5em}{0.5em}

% Set font family
\renewcommand{\familydefault}{\sfdefault}

% Reduce space between list items
\setlist[itemize]{itemsep=0pt, topsep=0pt}

% Begin document
\begin{document}

% Name and contact information
\begin{center}
    {\LARGE \textbf{Gustavo Gardusi}} \\ \vspace{0.5em}
    \href{mailto:gustavo.gardusi@gmail.com}{gustavo.gardusi@gmail.com} \textbar\ \href{https://github.com/gardusig}{github.com/gardusig}
\end{center}

% Experience
\section*{Work Experience}

\textbf{Amazon Web Services (AWS)}
    \hfill São Paulo, SP, Brazil 
    \\ Software Development Engineer II 
    \hfill \textit{Nov 2024 – Present}
\begin{itemize}
    \item Launched the Skill Builder public profile from Figma to production. Built a responsive UI using Next.js and Tailwind CSS, and implemented LinkedIn sharing with server-side meta tag injection across micro frontend boundaries.
    \item Extended an internal tool for the Training and Certification team to support async workflows and new in-game assets (NPC messages, diagrams, metadata). Built with React (CloudFront), Java/Python microservices (API Gateway, Lambda, DynamoDB), SQS-based processing, and CDK-managed infrastructure.
\end{itemize}

\textbf{Orkes}
    \hfill Cupertino, CA, USA (Remote from Brazil)
    \\ Software Engineer
    \hfill \textit{Jan 2022 – Jul 2023}
\begin{itemize}
    \item Created and maintained SDKs for Conductor — a microservice orchestration platform originally open-sourced by Netflix — in Python, Go, Java, C\#, and JavaScript. Delivered core features, tests, documentation, and usage examples.
    \item Worked directly with enterprise customers to design and implement SDK features, providing hands-on support to unblock the company’s largest client. Designed a batch-processing and parallel-worker model that improved task throughput and reduced average workflow latency by ~20\%.
\end{itemize}

\textbf{Amazon.com.br}
    \hfill São Paulo, SP, Brazil 
    \\ Software Development Engineer II 
    \hfill \textit{Apr 2021 – Jan 2022}
\begin{itemize}
    \item Enabled seller onboarding for Fulfillment by Amazon (FBA) by implementing state-specific invoicing rules across Java microservices. Delivered new APIs and extended existing ones to comply with Brazil’s diverse tax regulations.
    \item Integrated a load-testing step into the CI/CD pipeline to block underperforming builds, simulating traffic based on historical usage patterns. Distributed requests according to real production API hit ratios (e.g., 80\% concentrated on the top three endpoints), enabling realistic stress testing and accurate host scaling ahead of Black Friday 2021.
\end{itemize}

\textbf{Beyond}
    \hfill São Paulo, SP, Brazil 
    \\ Software Engineer
    \hfill \textit{Aug 2019 – Apr 2021}
\begin{itemize}
    \item Designed and implemented a high-frequency trading engine in C++ with Boost, colocated near the B3 stock exchange to minimize latency. Processed millions of daily orders using WebSocket and FIX, generating significant revenue through speed-sensitive strategies.
    \item Built a matching engine simulator that maintained a book of orders, executed trades, and reproduced exchange-like behavior for strategy debugging and validation. Developed Java APIs for ingesting market data and Python ETL pipelines that transformed it into datasets stored in S3. Enabled multi-day strategy simulations and generated performance reports with actionable recommendations based on historical results.
\end{itemize}

% Awards
\section*{Awards (Competitive Programming)}
\begin{itemize}
    \item \textbf{Meta HackerCup:} 3× Top 2,000 globally (T-shirt winner); best rank: 1,250th.
    \item \textbf{Codeforces / CodeChef:} First Division (Top 10\% globally).
    \item \textbf{ICPC:} 4× Latin America finalist. Best placement: 16th of 1,000+ teams.
\end{itemize}

% Education
\section*{Academic Background}

\textbf{? University} 
    \hfill Vancouver, BC, Canada (Remote)
    \\Bachelor of Science in Computer Science
    \hfill \textit{Expected 2025 – 2030 (in progress)}
\begin{itemize}
    \item Pursuing an accredited program while working full-time.
\end{itemize}

\end{document}
