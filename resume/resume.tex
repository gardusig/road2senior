\documentclass[a4paper,10pt]{article}
\usepackage[a4paper,margin=0.66in]{geometry}
\usepackage{enumitem}
\usepackage{sourcesanspro}
\usepackage{hyperref}
\usepackage{titlesec}
\usepackage{parskip}
\linespread{0.96}
\pagestyle{empty}

% Define title format
\titleformat{\section}{\Large\bfseries}{}{0em}{}[\titlerule]
\titlespacing*{\section}{0pt}{1.5em}{0.5em}

% Set font family
\renewcommand{\familydefault}{\sfdefault}

% Reduce space between list items
\setlist[itemize]{itemsep=0pt, topsep=0pt}

% Begin document
\begin{document}

% Name and contact information
\begin{center}
    {\LARGE \textbf{Gustavo Gardusi}} \\ \vspace{0.5em}
    \href{mailto:gustavo.gardusi@gmail.com}{gustavo.gardusi@gmail.com} \textbar\ \href{https://github.com/gardusig}{github.com/gardusig}
\end{center}

% Experience
\section*{Work Experience}

\textbf{Amazon Web Services (AWS)}
    \hfill São Paulo, SP, Brazil 
    \\ Software Development Engineer II 
    \hfill \textit{Nov 2024 – Present}
\begin{itemize}
    \item Launched the Skill Builder public profile from Figma to production. Built a responsive UI using Next.js and Tailwind CSS, and implemented LinkedIn sharing with server-side meta tag injection across micro frontend boundaries.
    \item Extended an internal tool for the Training and Certification team to support async workflows and new in-game assets (NPC messages, diagrams, metadata). Built with React (CloudFront), Java/Python microservices (API Gateway, Lambda, DynamoDB), SQS-based processing, and CDK-managed infrastructure.
\end{itemize}

\textbf{Orkes}
    \hfill Cupertino, CA, USA (Remote from Brazil)
    \\ Software Engineer
    \hfill \textit{Jan 2022 – Jul 2023}
\begin{itemize}
    \item Created and maintained SDKs for Conductor — a microservice orchestration platform originally open-sourced by Netflix — in Python, Go, Java, C\#, and JavaScript. Delivered core features, tests, documentation, and usage examples.
    \item Worked directly with enterprise customers to provide hands-on support. Designed a batch-processing and parallel-worker model that improved task throughput and reduced average workflow latency by ~20\%.
\end{itemize}

\textbf{Amazon.com.br}
    \hfill São Paulo, SP, Brazil 
    \\ Software Development Engineer II 
    \hfill \textit{Apr 2021 – Jan 2022}
\begin{itemize}
    \item Enabled seller onboarding for Fulfillment by Amazon (FBA) by implementing state-specific invoicing rules in existing Java microservices. 
    \item Extended APIs to comply with Brazil’s tax regulations. Integrated a load-testing step into the CI/CD pipelines of two services to block underperforming builds. Simulated traffic using real production API hit ratios, enabling accurate host scaling ahead of Black Friday 2021.
\end{itemize}

\textbf{Beyond}
    \hfill São Paulo, SP, Brazil 
    \\ Software Engineer
    \hfill \textit{Aug 2019 – Apr 2021}
\begin{itemize}
    \item Implemented a high-frequency trading engine in C++, colocated near the B3 stock exchange to minimize latency. Processed millions of daily orders using WebSocket and FIX, generating revenue through speed-sensitive strategies.
    \item Built a matching engine to reproduce the stock exchange behavior for trading strategy validation. Developed Java APIs for ingesting market data and Python ETL pipelines to transform it into datasets stored in S3. Enabled daily strategy simulations and generated performance reports with actionable recommendations based on historical results.
\end{itemize}

% Awards
\section*{Awards (Competitive Programming)}
\begin{itemize}
    \item \textbf{Meta HackerCup:} 3× Top 2,000 globally (T-shirt winner); best placement: 467th out of 30,000+ participants.
    \item \textbf{Codeforces / CodeChef:} Top 10\% globally (Candidate Master); ranked among 100,000+ users.
    \item \textbf{ICPC:} 4× Latin America Regional Finalist; best placement: 16th out of 1,000+ teams.
\end{itemize}

% Education
\section*{Academic Background}

\textbf{? University} 
    \hfill Vancouver, BC, Canada
    \\Bachelor of Science in Computer Science
    \hfill \textit{Expected 2025 – 2030 (in progress)}
\begin{itemize}
    \item Pursuing an accredited program while working full-time.
\end{itemize}

\end{document}
